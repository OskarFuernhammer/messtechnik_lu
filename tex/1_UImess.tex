\chapter{U/I/R-Messung und Messwerke}
\section{Einleitung}

\begin{table}[h]
	\centering
	\begin{tabular}{|c|c|}
		\hline 
		Teilnehmer 		& Oskar Fürnhammer, Katharina Kralicek, Patrick Mayr \\
		\hline 
		Datum 		& 26.11.2018 \\ 
		\hline 
		Messplatzbez. 	& CA0402-3 \\
		\hline
	\end{tabular} 
	\caption{Grundlegende Information der 1. Laborübung}
\end{table}

Verwendete Messgeräte:
\begin{itemize}
	\item{A}
	\item{B}
\end{itemize}


\begin{table}[h]
	\centering
	\begin{tabular}{ c | c }

Gerät				& Bezeichnung		\\
\hline

Multimeter			& Agilent U1232A 		\\
Multimeter			& Agilent U1232A 		\\
Multimeter			& Neumann 9140	 	\\
Netzgerät			& Rigol DP832 		\\
Oszilloskop			& Keysight DSOX2002A 	\\
Funktionsgenerator		& Agilent U1232A 		\\

	\end{tabular}

	\caption{Verwendete Geräte}

\end{table}

\newpage

\section{Spannungs-, Strom- und Widerstandsmessung}

Diese Übung setzt sich zusammen aus der XX

\subsection{Spannungsmessung}
\subsubsection{Berechnung des Innenwiderstandes des Voltmeters}
In der ersten Übung soll der Innenwiderstand des Voltmeters ermittelt werden. Die Schaltung zur Messung des Spannungswertes wird laut Abb. aufgebaut
~\\
TODO: SCHALTUNG		\\
~\\
Dieser Schaltung wurde mit einer Spannung $U$ von 10V versorgt. Der mit einem Ohmmeter gemessenen Widerstand $R_1$ ergab 100,2k$\Omega$. Die Spannung $U_V$ am Voltmeter betrug 9,90V.	\\
~\\
Die Spannung am Widerstand $R_1$ wird aus der Differenz von der Eingangsspannung und der Spannung am Voltmeter berechnet: 
\begin{center}
$U_{R1}$ = $U - U_V$ = 0,1V
\end{center}
Der Strom durch $R_1$ ergibt sich als Quotient aus der berechneten Spannung und dem Widerstand $R_1$:
\begin{center}
$I = \dfrac{U_{R1}}{R_1}$ = $\dfrac{U - U_V}{R_1}$ = $0,998\mu$A
\end{center}
Der Innenwiderstand des Voltmeters $R_i$ wird aus den zuvor berechneten Spannung und Strom berechnet:
\begin{center}
$R_i = \dfrac{U_V}{I}$ = 9,92M$\Omega$
\end{center}
\begin{table}[h]
	\centering
	\begin{tabular}{|c|c|}
	\hline 
	Spannungsquelle $U [V]$			& 10 		\\ 
	\hline 
	Vorwiderstand $R_1 [k\Omega]$		& 100,2	\\ 
	\hline 
	Spannung an Voltmeter $U_V [V]$ 	& 9,90	\\ 
	\hline 
	Spannung am Widerstand $R_1 [V]$	& 0,1		\\ 
	\hline 
	Strom durch $R_1 [\mu A]$		& 998		\\ 
	\hline 
	Innenwiderstand $R_i [M\Omega]$	& 9,92	\\ 
	\hline 
	\end{tabular}
	\caption{Auswertung dieser Übung}
\end{table}

Der Innenwiderstand des Multimeters ist im MegaOhm-Bereich, damit der Strom, der durch das Messgerät fließt, klein ist, um die Spannungsmessung so gering wie möglich zu verfälschen.

\subsubsection{Bestimmung der Spannung am Multimeter}
Bei dieser Messung soll eruriert werden, wie sich die Spannungsmessung auf die Berechnung des Innenwiderstandes auswirkt.
~\\
TODO: SCHALTUNG		\\
~\\
Dieser Schaltung wurde mit einer Spannung $U$ von 10V versorgt. Der mit einem Ohmmeter gemessenen Widerstand $R_M$ ergab 100,2k$\Omega$. Die Spannungen $U_{V1}$ und $U_{V2}$ an den beiden Voltmetern betrugen 9,80V.	\\
~\\
Die beiden parallel geschalteten Voltmetern werden in der Berechnung als Ersatz-Innenwiderstände ersetzt. Es wird angenommen, dass beide Ersatzwiderstände den selben Wert von 9,92$M\Omega$ haben.
\begin{center}
$R_{iG} = \dfrac{R_{i1} \cdot R_{i2}}{R_{i1} + R_{i2}}$
\end{center}
Die Spannung am Voltmeter V2 wird als Teilspannung mit Hilfe der Spannungsteilerregel berechnet:
\begin{center}
$U_{V2} = U \cdot \dfrac{R_{iG}}{R_M + R_{iG}}$
\end{center}
\begin{table}[h]
	\centering
	\begin{tabular}{|c|c|}
	\hline 
	Spannungsquelle $U [V]$			& 10 		\\ 
	\hline 
	Spannung an Voltmeter $U_{V1} [V]$ 	& 9,80	\\ 
	\hline 
	Spannung an Voltmeter $U_{V2} [V]$ 	& 9,80	\\ 
	\hline 
	\end{tabular}
	\caption{Auswertung dieser Übung}
\end{table}
TODO: Interpretation

\subsubsection{Messbereichserweiterung}
Die Destination dieser Teilübung ist, den Messbereichserweiterung der Spannungsmessung zu erstellen.
~\\
TODO: SCHALTUNG		\\
~\\
Dieser Schaltung wurde mit einer Spannung $U$ von 10V versorgt. Die Widerstände $R_M$ und $R_V$ wurde jeweils mit einem Ohmmeter gemessen und ergaben 99,5k$\Omega$ für $R_M$ bzw. 100,2k$\Omega$ für $R_V$. Die Spannung $U_{RM}$ an dem Widerstand $R_M$ betrug 4,952V.	\\
~\\
Der Faktor $f_{ME}$ der Messbereichserweiterung wird berechnet aus:
\begin{center}
$f_{ME} = \dfrac{U}{U_{RM}}$
\end{center}
Die Eingangsspannung $U$ wird aus der Addition von der Spannung vom Voltmeter und vom 
\begin{center}
$U = U_V + U_{RM} = I \cdot R = I \cdot (R_V + R_M \parallel R_{i2})$
\end{center}
Die Spannung lässt durch X berechnen.
\begin{center}
$U_V = I \cdot R_V$
\end{center}
\begin{center}
$U_{RM} = I \cdot (R_M \parallel R_{i2})$
\end{center}
Das Voltmeter wird durch eine Ersatz-Innenwiderstand ersetzt, somit ergibt sich der Parallelwiderstand $R_P$
\begin{center}
$R_P = \dfrac{R_M \cdot R_{i2}}{R_M + R_{i2}}$
\end{center}
Setzt man die obrigen Formeln in die Formel ein, bekommt man:
\begin{center}
$f_{ME} = \dfrac{R_V + R_P}{R_P}$
\end{center}
\begin{table}[h]
	\centering
	\begin{tabular}{|c|c|}
	\hline 
	Spannungsquelle $U [V]$					& 10 		\\ 
	\hline 
	Spannung am Widerstand $R_M$ [V] 			& 4,952	\\ 
	\hline 
	Widerstand $R_M [k\Omega]$ 				& 99,5	\\ 
	\hline 
	Widerstand $R_V [k\Omega]$ 				& 100,2	\\ 
	\hline 
	Faktor der Messbereichserweiterung $f_{ME}$	& XXX	\\ 
	\hline 
	\end{tabular}
	\caption{Auswertung dieser Übung}
\end{table}
TODO: Interpretation


\subsection{Strommessung}
\subsubsection{Berechnung des Innenwiderstandes des Amperemeters}
In der ersten Übung soll der Innenwiderstand des Amperemeters ermittelt werden. Die Schaltung zur Messung des Stromwertes wird folgende Schaltung aufgebaut.
~\\
TODO: SCHALTUNG		\\
~\\
Um dieser Schaltung mit einem Strom $I$ von $500,6\mu A$ zu versorgen, wurde die Spannungsquelle $U_A$ auf 1,501V gestellt. Der mit einem Ohmmeter gemessenen Widerstand $R_1$ ergab 4,63k$\Omega$. \\
~\\
Der Innenwiderstand des Amperemeters $R_i$ wird aus dem gemessenen Spannungs- und Stromwert berechnet:
\begin{center}
$R_i = \dfrac{U_A}{I}$ = 3k$\Omega$
\end{center}
\begin{table}[h]
	\centering
	\begin{tabular}{|c|c|}
	\hline 
	Spannung am Voltmeter $U [V]$				& 1,501	\\ 
	\hline 
	Strom durch das Amperemeter $I [\mu A]$ 		& 500,6	\\ 
	\hline 
	Innenwiderstand $R_i [k\Omega]$			& 4,633	\\ 
	\hline 
	\end{tabular}
	\caption{Auswertung dieser Übung}
\end{table}

\subsubsection{Bestimmung des Stromes durch das Amperemeter}
Bei dieser Messung soll eruriert werden, wie sich die Strommessung auf die Berechnung des Innenwiderstandes auswirkt.
~\\
TODO: SCHALTUNG		\\
~\\
Die Spannungsversorgung wurde so lange erhöht, bis der Strom $I_1$ $500\mu A$ erreicht wurden. Danach wurde der Bügel entfernt und erneut den Strom $I_2$ abgelesen. Die Ströme wurden mit dem Amperemeter $A_1$ gemessen.
Die Spannung am Widerstand $R_1$ wird aus der Differenz von der Eingangsspannung und der Spannung am Voltmeter berechnet:
\begin{center}
$U_{R1}$ = $U - U_V$ = 0,1V
\end{center}
Der Strom durch $R_1$ ergibt sich als Quotient aus der berechneten Spannung und dem Widerstand $R_1$:
\begin{center}
$I = \dfrac{U_{R1}}{R_1}$ = $\dfrac{U - U_V}{R_1}$ = $0,998\mu$A
\end{center}
Der Innenwiderstand des Voltmeters $R_i$ wird aus den zuvor berechneten Spannung und Strom berechnet:
\begin{center}
$R_i = \dfrac{U_V}{I}$ = 9,92M$\Omega$
\end{center}
\begin{table}[h]
	\centering
	\begin{tabular}{|c|c|}
	\hline 
	Spannungsquelle $U [V]$			& 10 		\\ 
	\hline 
	Vorwiderstand $R_1 [k\Omega]$		& 100,2	\\ 
	\hline 
	Spannung an Voltmeter $U_V [V]$ 	& 9,90	\\ 
	\hline 
	Spannung am Widerstand $R_1 [V]$	& 0,1		\\ 
	\hline 
	Strom durch $R_1 [\mu A]$		& 998		\\ 
	\hline 
	Innenwiderstand $R_i [M\Omega]$	& 9,92	\\ 
	\hline 
	\end{tabular}
	\caption{Auswertung dieser Übung}
\end{table}


