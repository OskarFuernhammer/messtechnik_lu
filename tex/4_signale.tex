\chapter{Signalübertragung}

\section{Einleitung}

\begin{table}[!h]
	\centering
	\begin{tabular}{|c|c|}
		\hline 
		Teilnehmer 		& Oskar Fürnhammer, Katharina Kralicek, Patrick Mayr \\
		\hline 
		Datum 		& 03.12.2018 \\ 
		\hline 
		Messplatzbez. 	& CA0406-1 \\
		\hline
	\end{tabular} 
	%\caption{Grundlegende Information der 1. Laborübung}
\end{table}

\begin{table}[!h]
	\centering
	\begin{tabular}{ c | c }

Gerät				& Bezeichnung		\\
\hline

Multimeter			& Neumann 9400				\\
Multimeter			& Agilent U1232A				\\
Netzgerät			& Rigol DP832 				\\
Oszilloskop			& Agilent Technologies DSO-X 2002A 	\\
Funktionsgenerator		& Agilent 33500B 				\\

	\end{tabular}

	\caption{Verwendete Geräte}
\end{table}


\newpage

\section{Signalübertragung auf ungestörten Drahtleitungen}

\subsection{Reflexionen}

In dieser Übung soll analysiert werden, welchen Einfluss der Abschlusswiderstand auf die Signalübertragung auf einer 30m langen Drahtleitungen haben. Um die Reflexion am Abschlusswiderstand messen zu können, wurden die Leitung an den Abschlüssen der entsprechenden Kabelboards angeschlossen. Die nicht genutzten Abschlüssen wurden mit einem Kurzschlusswiderstand versehen. Als Signalübertragung wurde die Single-Ended Voltage Signalling genommen. Am Eingang wurde der Frequenzgenerator angeschlossen und die entsprechenden Einstellungen für das Signal sind in der Tabelle \ref{tb:einst_refl} ersichtlich. \\

\begin{table}[!h]
	\centering
	\begin{tabular}{|c|c|c|c|}
		\hline 
		Amplitude		& Offset	& Frequenz	& Signalform		\\ 
		\hline 
		100$mV_{pp}$	& 4V		& 10kHz	& Rechteck		\\ 
		\hline 
	\end{tabular}
	\caption{Einstellungen des Frequenzgenerators}
	\label{tb:einst_refl}
\end{table}
~\\
Es wurde die Reflexionen bei verschiedenen Abschlusswiderstand aufgenommen. \\

\begin{table}[!h]
	\centering
	\begin{tabular}{|c|c|c|c|}
	\hline 
	Abschlusswiderstand $R_A$	& $|V_{1.peak}| [mV]$	& $|V_{2.peak}| [mV]$	& $\Delta t_{peaks} [ns]$		\\ 
	\hline 
	$\infty$				& 135				& 84,75			& 370		\\ 
	\hline 
	50					& 25,5			& 8,5				& 350		\\ 
	\hline 
	100					& -				& -				& -		\\ 
	\hline 
	200					& 33,25			& 8,5				& 380		\\ 
	\hline 
	\end{tabular}
	\caption{Auswertung dieser Übung}
	\label{tb:abschlusswid}
\end{table}
~\\
Die Dämpfung kann über das Verhältnis von U(100) zu U(0) berechnet werden.
\begin{equation}
	\dfrac{U(100)}{U(0)}
	\label{eq:dampf}
\end{equation}
\begin{equation}
	U(z) = U(0) \cdot e^{-\alpha z}
	\label{eq:dampf_u}
\end{equation}
~\\
Das $\alpha$ wird mit der Formel \ref{eq:dampf_u} berechnet. Für z wird die doppelte Kabellänge eingesetzt und die Spannungswerte für totale Reflexion werden aus der Tabelle \ref{tb:abschlusswid} entnommen. Daraus ergibt für $\alpha$ den Wert $7,76\cdot 10^{-3} \dfrac{1}{m}$.
\begin{equation}
	\alpha = -\dfrac{1}{60} \cdot ln\bigg(\dfrac{U(60)}{U(0)}\bigg) = 7,76\cdot 10^{-3} \dfrac{1}{m}
\end{equation}
~\\
Zur Berechnung der Dämpfung setzt man den zuvor berechneten Wert von $\alpha$ und die Formel \ref{eq:dampf_u} in die Formel für die Dämpfung ein und das ergibt:
\begin{equation}
	\text{Dämpfung} = 20 log\bigg(\dfrac{U(100)}{U(0)}\bigg) =  20 log(e^{-\alpha 100}) = -6,745 dB / 100m
	\label{eq:dampf}
\end{equation}

\begin{table}[!h]
	\centering
	\begin{tabular}{|c|}
		\hline 
		Dämpfung [$dB / 100m$]		\\ 
		\hline 
		-6,745 	\\ 
		\hline 
	\end{tabular}
\end{table}
~\\
Bei der Messung mit dem Abschlusswiderstand von $100 \Omega$ könnten keine sinnvolle Daten aufgenommen, da der Abschlusswiderstand gleich dem Wellenwiderstand ist. Daher treten auch keine Reflexionen auf.
~\\
Die Geschwindigkeit c des Signals wird über die Formel \ref{eq:speed} berechnet und daraus ergibt sich für eine Signallaufzeit von 370ns eine Geschwindigkeit von $162\cdot 10^{6} \dfrac{m}{s}$. Das ist in etwa $\dfrac{2}{3}$ der Lichtgeschwindigkeit.
\begin{equation}
	c = \dfrac{2 \cdot l_{Kabel}}{\Delta t}
	\label{eq:speed}
\end{equation}

\begin{table}[!h]
	\centering
	\begin{tabular}{|c|}
		\hline 
		Leitungswiderstand $Z_L$ [$\Omega$]		\\ 
		\hline 
		100	\\ 
		\hline 
	\end{tabular}
\end{table}
\begin{table}[!h]
	\centering
	\begin{tabular}{|c|}
		\hline 
		Ausbreitungsgeschwindigkeit c [$\dfrac{m}{s}$]		\\ 
		\hline 
		$162\cdot 10^{6}$		\\ 
		\hline 
	\end{tabular}
\end{table}


\subsection{Bandbreite}

Die Bandbreite ist ein Bereich im Frequenzspektrum, in dem die Verstärkung größer als 3db ist. Um die Grenzfrequenz zu ermitteln, wird sukzessive die Frequenz erhöht, bis die Ausgangsgröße um das $\dfrac{1}{\sqrt{2}}$-fache kleiner als die Eingangsgröße ist. Am Eingang wurde der Frequenzgenerator angeschlossen und die entsprechenden Einstellungen für das Signal sind in der Tabelle \ref{tb:einst_band} ersichtlich. Das Kabel wird mit dem Abschlusswiderstand $R_A$ = 100 $\Omega$ abgeschlossen.

\begin{table}[!h]
	\centering
	\begin{tabular}{|c|c|c|c|}
		\hline 
		Amplitude		& Offset	& Frequenz		& Signalform		\\ 
		\hline 
		100$mV_{pp}$	& 4V		& 100kHz-10MHz	& Rechteck		\\ 
		\hline 
	\end{tabular}
	\caption{Einstellungen des Frequenzgenerators}
	\label{tb:einst_bandl}
\end{table}
~\\
Die Übertragungsstrecke zeigt ein Tiefpassverhalten.

\begin{table}[!h]
	\centering
	\begin{tabular}{|c|}
		\hline 
		Bandbreite B [MHz]		\\ 
		\hline 
		6,9				\\ 
		\hline 
	\end{tabular}
\end{table}



\section{Signalübertragung über einen gestörten Kanal}

\subsection{Bestimmung des Signal-to-Noise Ratios}

In dieser Übung soll evaluiert werden, wie sich die Störsignalen auf die Signalübertragung auswirken. Die Schaltung wird um den Störsender erweitert, der parallel zur Signalleitung angebracht wird. Am Eingang wurde der Frequenzgenerator angeschlossen und die entsprechenden Einstellungen für das Signal sind in der Tabelle \ref{tb:einst_stor} ersichtlich. Das Kabel wird mit dem Abschlusswiderstand $R_A$ = 100 $\Omega$ abgeschlossen.

\begin{table}[!h]
	\centering
	\begin{tabular}{|c|c|c|c|c|}
	\hline 
			& Amplitude		& Offset 	& Frequenz 		& Signalform		\\ 
	\hline 
	Sender	& -			& 4V		& -			& DC			\\ 
	\hline 
	Störsender	& 3Vpp		& 2V		& 10MHz		& Noise 		\\ 
	\hline 
	\end{tabular}
	\caption{Einstellungen des Frequenzsgenerators}
	\label{tb:einst_stor}
\end{table}
~\\
Bei der Messung am Oszilloskop ergaben die folgenden Werten:

\begin{table}[!h]
	\centering
	\begin{tabular}{|c|}
		\hline 
		Rauschen $U_{n, se}$ [mV]		\\ 
		\hline 
		13,7	\\ 
		\hline 
	\end{tabular}
\end{table}
~\\
Die Amplitude wird mit der Formel \ref{eq:snr_ampl} berechnet, mit der Bedingung, dass das SNR 0dB beträgt. Daraus ergibt sich, dass die Spannung des Signals $U_{Signal}$ gleich der Spannung des Rauschens $U_{Rauschen}$ ist.
\begin{equation}
	U_s(30) = U(0) \cdot e^{-\alpha 30} \cdot \sqrt{2} = 15,35mV
	\label{eq:snr_ampl}
\end{equation}

\begin{table}[!h]
	\centering
	\begin{tabular}{|c|}
		\hline 
		Amplitude $U_s$ [mV]		\\ 
		\hline 
		15,35		\\ 
		\hline 
	\end{tabular}
\end{table}
~\\
Um das SNR zu verbessern, ohne die Signalstärke erhöhen zu müssen, kann man einen Tiefpass-Filter am Eingang einbauen oder die Signalübertragung von Single-Ended Voltage Signalling auf die differentielle Übertragung umbauen.
~\\
Bei der nächsten Aufgabe wird ein Filter am Eingang mit dem Umstecken eines Jumpers dazu geschaltet.

\begin{table}[!h]
	\centering
	\begin{tabular}{|c|c|c|}
	\hline 
	Filtertype	& Ordnung		& Grenzfrequenz $f_c$ [kHz]		\\ 
	\hline 
	Tiefpass	& erste		& 180					\\ 
	\hline 
	\end{tabular}
	\caption{Auswertung dieser Übung}
\end{table}
~\\

\begin{equation}
	SNR = 20 \cdot log\bigg(\dfrac{U_{Signal}}{U_{Rauschen}}\bigg)
	\label{eq:snr_ampl}
\end{equation}
\begin{table}[!h]
	\centering
	\begin{tabular}{|c|}
		\hline 
		SNR [dB]		\\ 
		\hline 
		72		\\ 
		\hline 
	\end{tabular}
\end{table}

\begin{table}[!h]
	\centering
	\begin{tabular}{|c|}
		\hline 
		Bandbreite B [MHz]		\\ 
		\hline 
		XXX		\\ 
		\hline 
	\end{tabular}
\end{table}



\subsection{Differentielle Übertragung}
Um das Verhalten einer differentiellen Übertragung zu untersuchen, wurde der Aufbau mit den entsprechenden Sende- und Empfangsmodulen von einer Single-Ended auf eine differentielle Übertragung umgebaut. Am Eingang wurde der Frequenzgenerator angeschlossen und die entsprechenden Einstellungen für das Signal sind in der Tabelle \ref{tb:einst_diff} ersichtlich. Das Kabel wird mit dem Abschlusswiderstand $R_A$ = 100 $\Omega$ abgeschlossen.


\begin{table}[!h]
	\centering
	\begin{tabular}{|c|c|c|c|c|}
	\hline 
			& Amplitude		& Offset 	& Frequenz 		& Signalform		\\ 
	\hline 
	Sender	& -			& 4V		& -			& DC			\\ 
	\hline 
	Störsender	& 3Vpp		& 2V		& 10MHz		& Noise 		\\ 
	\hline 
	\end{tabular}
	\caption{Einstellungen des Frequenzsgenerators}
	\label{tb:einst_diff}
\end{table}
~\\
FRAGEN beantworten!!

\begin{table}[!h]
	\centering
	\begin{tabular}{|c|}
		\hline 
		Rauschen $U_{n, de}$ [mV]		\\ 
		\hline 
		2,5	\\ 
		\hline 
	\end{tabular}
\end{table}
~\\
Die Amplitude wird mit der Formel \ref{eq:diff_ampl} berechnet:
\begin{equation}
	U_s(30) = U(0) \cdot e^{-\alpha 30} \cdot \sqrt{2} = 2,8mV
	\label{eq:diff_ampl}
\end{equation}
\begin{table}[!h]
	\centering
	\begin{tabular}{|c|}
		\hline 
		Amplitude $U_s$ [V]		\\ 
		\hline 
		2,8		\\ 
		\hline 
	\end{tabular}
\end{table}

\begin{table}[!h]
	\centering
	\begin{tabular}{|c|}
		\hline 
		$U_{n, se} / U_{n, de}$ [1]		\\ 
		\hline 
		5,48		\\ 
		\hline 
	\end{tabular}
\end{table}

\subsection{Signalübertragung von digitalen Signalen}

Möchte man ein 10kHz-Sinus-Signal mit einer Abtastrate von 100kSPS mit einer Genauigkeit von 16-bit abtasten, benötigt man eine Bitrate BR = 1,6 Mbps.

\begin{table}[!h]
	\centering
	\begin{tabular}{|c|}
		\hline 
		Bitrate BR [Mbps]		\\ 
		\hline 
		1,6		\\ 
		\hline 
	\end{tabular}
\end{table}

In dieser Übung wird ein Low-Voltage Differential Signaling (LVDS) als Übertragungsstrecke verwendet. Am Eingang wurde der Frequenzgenerator angeschlossen und die entsprechenden Einstellungen für das Signal sind in der Tabelle \ref{tb:einst_lvds} ersichtlich. Dabei wird eine Pseudo Random Bit Sequence (PRBS) als Eingangssignal verwendet, dieses generiert zufällig die Bitfolge.

\begin{table}[!h]
	\centering
	\begin{tabular}{|c|c|c|c|c|}
	\hline 
			& Amplitude		& Offset 	& Frequenz 		& Signalform		\\ 
	\hline 
	Sender	& 3Vpp		& 1,5V	& 1,6MHz		& PRBS		\\ 
	\hline 
	Störsender	& 3Vpp		& 2V		& 10MHz		& Noise 		\\ 
	\hline 
	\end{tabular}
	\caption{Einstellungen des Frequenzsgenerators}
	\label{tb:einst_lvds}
\end{table}

Um ein stehendes Bild am Oszilloskop zu bekommen, triggert man das Signal bei steigender und fallender Flanke. In der Tabelle \ref{tb:auswertung_lvds} wurden die von dem Bild abgelesene Werte eingetragen.

\begin{table}[!h]
	\centering
	\begin{tabular}{|c|c|}
	\hline 
	Bitrate [Mbps]							& 1,6		\\ 
	\hline 
	$U_1$ Spannungslevel bei digital 1 [mV]				& 394		\\ 
	\hline 
	$U_0$ Spannungslevel bei digital 0 [mV]				& -419	\\ 
	\hline 
	Anstiegszeit [ns]							& 50		\\ 
	\hline 
	Abfallzeit [ns]							& 68		\\ 
	\hline 
	Jitter des Übergangszeitpunktes 0-1 und 1-0 [ns]		& 35		\\ 
	\hline 
	$\sigma_1$ Rauschen des Spannungslevels von 1 [mV]	& 43,5	\\ 
	\hline 
	$\sigma_0$ Rauschen des Spannungslevels von 0 [mV]	& 46,5	\\ 
	\hline 
	\end{tabular}
	\caption{Auswertung}
	\label{tb:auswertung_lvds}
\end{table}
~\\
Die Bit Error Rate (BER) wird über diese Formel berechnet.
\begin{equation}
	Bit Error Rate (BER) = erfc\bigg(\dfrac{Q}{\sqrt{2}}\bigg)
\end{equation}
\begin{equation}
	Bitfehlerparameter (Q) = \dfrac{U_1 - U_0}{\sigma_1 + \sigma_0}
\end{equation}
~\\
Für eine Bit-Error-Rate von $10^{-9}$ ist ein Signal-to-Noise Rate von 21,6 dB nötig.
\begin{equation}
	SNR = 20 \cdot log\bigg(\dfrac{U_1 - U_0}{U_n}\bigg) -> U_n = \dfrac{U_1- U_0}{10^{\dfrac{SNR}{20}}}
\end{equation}
\begin{table}[!h]
	\centering
	\begin{tabular}{|c|}
		\hline 
		Rauschen $U_{n, BER = 10^{-9}}$ [mV]		\\ 
		\hline 
		67,6		\\ 
		\hline 
	\end{tabular}
\end{table}



\section{Signalübertragung mittels Stromsignalen}
Bei Spannungssignalen ist die Signalamplitude am Ausgang abhängig vom Serienwiderstand der Leitung. Bei Stromsignalen passt der Sender die Spannung so an, dass der entsprechende Strom fließt. Am Eingang wurde der Frequenzgenerator angeschlossen und die entsprechenden Einstellungen für das Signal sind in der Tabelle \ref{tb:einst_strom} ersichtlich. Der Störsender wird bei dieser Übung nicht mehr benötigt.

\begin{table}[!h]
	\centering
	\begin{tabular}{|c|c|c|c|c|}
	\hline 
			& Amplitude		& Offset 	& Frequenz 		& Signalform		\\ 
	\hline 
	Sender	& 5Vpp		& 2,5V	& 10kHz		& Sinus		\\ 
	\hline 
	\end{tabular}
	\caption{Einstellungen des Frequenzsgenerators}
	\label{tb:einst_strom}
\end{table}

Für die Berechnung der Sensitivität wird die Formel \ref{eq:sens} herangezogen.
\begin{equation}
	S = \dfrac{U_1 - U_2}{R}
	\label{eq:sens}
\end{equation}

Der Versuch wurde mit einem Stromsender gestartet und folgende Sensitivität wurde berechnet.
\begin{equation}
	S = \dfrac{550mV - 560mV}{96\Omega}
	\label{eq:sens}
\end{equation}

\begin{table}[!h]
	\centering
	\begin{tabular}{|c|}
		\hline 
		Sensitivität [$\mu V / \Omega$]		\\ 
		\hline 
		-104		\\ 
		\hline 
	\end{tabular}
\end{table}

Dann wurde der Stromsender mit einem Spannungssender Single-Ended getauscht und man kam auf einer höhere Sensitivität.
\begin{equation}
	S = \dfrac{1,44V - 2,51V}{96\Omega}
	\label{eq:sens}
\end{equation}

\begin{table}[!h]
	\centering
	\begin{tabular}{|c|}
		\hline 
		Sensitivität [$mV / \Omega$]		\\ 
		\hline 
		-11,15		\\ 
		\hline 
	\end{tabular}
\end{table}





