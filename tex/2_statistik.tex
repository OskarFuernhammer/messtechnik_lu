\chapter{Statistik und Leistungsmessung}
\section{Einleitung}
\begin{table}[]
	\centering
	\begin{tabular}{|c|c|}
		\hline 
		Teilübung 	& Statistik und Leistungsmessung \\
		\hline 
		Teilübungsnr. 		& 2	 \\ 
		\hline 
		Datum 		& 28.11.2018 \\ 
		\hline 
		Messplatzbez. 	& CA \\
		\hline
	\end{tabular} 
	\caption{Grundlegende Information der 2. Laborübung}
\end{table}

Verwendete Messgeräte:
\begin{itemize}
	\item{A}
	\item{B}
\end{itemize}

Im Rahmen der 2. Laborübung sollten fünf unterschiedliche Impedanzen ($Z1$ - $Z5$) vermessen werden. Dabei war lediglich deren Struktur (siehe Tabelle XXXX) im Vorhinein bekannt. 

\section{Strommessung}
TO DO
\section{Widerstandsmessung}
TO DO
\section{Impedanzmessung}
TO DO
\section{Fehlerforpflanzung}
TO DO
\section{Impedazmessung mit LCR-Meter}
TO DO
\section{5/8-Methode}
TO DO
\section{Leistungsmessung}