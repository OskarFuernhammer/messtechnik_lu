\chapter{Statistik und Leistungsmessung}
\section{Einleitung}
\begin{table}[h]
	\centering
	\begin{tabular}{|c|c|}
		\hline 
		Teilübung 	& Statistik und Leistungsmessung \\
		\hline 
		Teilübungsnr. 		& 2	 \\ 
		\hline 
		Datum 		& 28.11.2018 \\ 
		\hline 
		Messplatzbez. 	& CA \\
		\hline
	\end{tabular} 
	\caption{Grundlegende Information der 2. Laborübung}
\end{table}
\noindent
Im Rahmen der 2. Laborübung sollten fünf unterschiedliche Impedanzen ($Z1$ - $Z5$) vermessen werden. Dabei war lediglich deren Struktur (siehe Tabelle \ref{tb:imp}!) im Vorhinein bekannt. Es wurde zuerst ein passender Strommessshunt ausgewählt und die Schaltung konzipiert. Um aus den erhaltenen Spannungswerten den dazugehörigen Strom bestimmen zu können ist natürlich die genau Kenntnis über den Widerstandswert unabdingbar, weshalb dieser zu Beginn mehrmals und mit unterschiedlichen Methoden bestimmt worden ist. Die eigentliche Impedanzmessung wurde darauf hin mit einem analogen Oszilloskop durchgeführt. \\
Alle dabei verwendenten Messgeräte sind in Tabelle \ref{tb:messgeraete} aufgelistet.
\begin{table}[h]
	\begin{tabular}{|c|c|}
	\hline 
	Gerät & Bezeichnung \\ 
	\hline 
	Handmultimeter & Agilent U1232A \\ 
	\hline 
	Handmultimeter & Mastech MS8221C \\ 
	\hline 
	Handmultimeter & Neumann 9140 \\ 
	\hline 
	Desktopmultimeter & Agilent 34461A \\ 
	\hline 
	Analoges Oszilloskop & XXXX-------XXX DS-6612 \\ 
	\hline 
	\end{tabular} 
	\centering
	\caption{Verwendete Messgeräte}
	\label{tb:messgeraete}
\end{table}

\subsubsection{Teilnahmebestätigung}


Hiermit XXX \\
XX

\section{Strommessung}
Um den Strom durch einen bestimmten Strang zu messen musste zuerst eine passende Schaltung entworfen bzw. in weiterer Folge ein passender Messshunt ausgewählt werden. Unter der Bedingung, dass bei einer Eingangsspannung von $U=10\,V_{\text{pp}}$ ein maximaler Strom von $I_{\text{max}}=5\,$mA nicht überschritten werden soll, ergibt sich mit dem Ohm'schen Gesetzt direkt
\begin{equation}
	R_{i,\text{min}} = \frac{\hat{U}}{I_{max}} = \frac{5\,\text{V}}{5\,\text{mA}} = 1\,\text{k}\Omega
	\label{eq:widerstandsdim}
\end{equation}
\section{Widerstandsmessung}
Da 
\begin{table}[h]
	\centering
	\begin{tabular}{|l|c|c|}
	\hline 
	Messung & Widerstandswert $R_i$ [$\Omega$]\\ 
	\hline 
	M1 - Agilent U1232A & 986		\\ 
	\hline 
	M2 - Mastech MS8221C & 984		\\ 
	\hline 
	M3 - Neumann 9140 & 988		\\ 
	\hline 
	DM - Agilent 34461A & 987		\\ 
	\hline 
	\end{tabular}
	\caption{Gemessene Widerstandswerte}
	\label{tb:widerstandswerte}
\end{table} \noindent
Damit ergibt sich der Mittelwert zu
\begin{equation}
	\overline{R_i} = \frac{1}{N} \sum\limits_{j=0}^N R_{i,j} = 
	\label{eq:mittelw}
\end{equation}
Die empirische Standardabweichung wurde wiederum folgendermaßen berechnet:
\begin{equation}
	s(\overline{R_i}) = \sqrt{\frac{1}{N-1} \sum\limits_{j=0}^N (R_{i,j} - \overline{R_i})^2}= 
	\label{eq:stdabw}
\end{equation}
Das Desktopmultimeter bietet die Funktion diverse statistische Größen direkt zu berechnen. Es hat sich gezeigt, dass mit zunehmender Aperaturbreite die Werte annährend gaußverteilt erscheinen. Auch der Effekt der PLC (Power Line Cycles) wurde untersucht. Wobei festgestellt worden ist, dass der Widerstandswert, höchst wahrscheinlich auf Grund der Temperaturabhngigkeit, bei langen Messzeiten stark zu driften beginnt. Die erhaltenen Messdaten sind in Tabelle \ref{tb:widerstand_dm} zusammengefasst.
%\begin{figure}
%	\includegraphics[scale=•]{•}
%\end{figure}
\begin{table}[h]
	\centering
	\begin{tabular}{|c|c|c|c|}
	\hline 
	PLC & Samples & Mittelwert [$\Omega$] & Standardabweichung [m$\Omega$] \\ 
	\hline 
	$0.02$ & 15k & 987.44 & 20 \\ 
	\hline 
	$0.2$ & 15k & 987.421 & 12 \\ 
	\hline 
	$1$ & 273 & 987.442 & 3 \\ 
	\hline 
	1 & 1017 & 987.416 & 2 \\ 
	\hline 
	1 & 5456 & 987.401 & 13 \\ 
	\hline 
	\end{tabular}
	\caption{Widerstandsmessung mit dem Desktopmultimeter}
	\label{tb:widerstand_dm}
\end{table}

\section{Impedanzmessung}
Um die unbekannten Impedanzen zu bestimmen werden Spannung und Strom (über Spannungsabfall an $R_i$) mit einem analogen Kathodenstrahloszilloskop gemessen. Dazu wird mittels Funktionsgenerator ein Sinus mit Amplitude 5V (10Vpp) angelegt. Durch die Phasenverschiebung und Amplitude des Stroms bei verschiedenen Frequenzen kann auf die Struktur sowie die Größe der Impedanz geschlossen werden. \\
XX \\
Formeln \\
\begin{table}[h]
	\centering
	\begin{tabular}{|c|c|c|c|c|c|}
	\hline 
	Strang & f [kHz] & u [V] & i [A] & Z [$\Omega$] & Struktur \\ 
	\hline 
	S1 & 1 & • & • & • & • \\ 
	\hline 
	• & 15 & • & • & • & • \\ 
	\hline 
	S2 & 1 & • & • & • & • \\ 
	\hline 
	• & 15 & • & • & • & • \\ 
	\hline 
	S3 & 1 & • & • & • & • \\ 
	\hline 
	• & 15 & • & • & • & • \\ 
	\hline 
	S4 & 1 & • & • & • & • \\ 
	\hline 
	• & 15 & • & • & • & • \\ 
	\hline 
	S5 & 1 & • & • & • & • \\ 
	\hline 
	• & 15 & • & • & • & • \\ 
	\hline 
	\end{tabular}
	\caption{•}
	\label{tb:imp}
\end{table}
\section{Fehlerforpflanzung}
\begin{table}
	\centering
	\begin{tabular}{|c||c|c|}
	\hline 
	Messung Nr. & $x_1 = I_{RMS}$ [V] & $x_2 = \Phi$ [rad] \\ 
	\hline 
	1 & • & • \\ 
	\hline 
	2 & • & • \\ 
	\hline 
	3 & • & • \\ 
	\hline 
	4 & • & • \\ 
	\hline 
	5 & • & • \\ 
	\hline 
	6 & • & • \\ 
	\hline 
	$\overline{x_i}$ & • & • \\ 
	\hline 
	$s(\overline{x_i})$ & • & • \\ 
	\hline 
	$\frac{\partial P}{\partial x_i}$ & • & • \\ 
	\hline 
	$(\frac{\partial P}{\partial x_i})^2 s^2(\overline{x_i})$ & • & • \\ 
	\hline 
	Kovarianz & \multicolumn{2}{c|}{•} \\ 
	\hline 
	$s (\overline{P})$ & \multicolumn{2}{c|}{•} \\ 
	\hline 
	\end{tabular} 
	\caption{•}
\end{table}
\section{Impedazmessung mit LCR-Meter}
\begin{table}[h]
	\centering
	\begin{tabular}{|c||c|c|c|}
	\hline 
	Strang & C/L [nF/mH] & R [$\Omega$] & Z [$\Omega$] \\ 
	\hline 
	S1 & 47.84 & 13.57 & 300.9 \\ 
	\hline 
	S2 & 1.1018 & 17.80 & 18.91 \\ 
	\hline 
	S3 & 97.89 & \num{2.701e+3} & \num{3.153e+3} \\ 
	\hline 
	S4 & 102 & 27.5 & \num{1.56e+3} \\ 
	\hline 
	S5 &  & \num{8.066} & \num{8.066e+3} \\ 
	\hline 
	\end{tabular} 
\end{table}
TO DO
\section{5/8-Methode}
TO DO
\section{Leistungsmessung}
