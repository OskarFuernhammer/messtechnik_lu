\chapter{Statistik und Leistungsmessung}
\section{Einleitung}
\begin{table}[]
	\centering
	\begin{tabular}{|c|c|}
		\hline 
		Teilübung 	& Statistik und Leistungsmessung \\
		\hline 
		Teilübungsnr. 		& 2	 \\ 
		\hline 
		Datum 		& 28.11.2018 \\ 
		\hline 
		Messplatzbez. 	& CA \\
		\hline
	\end{tabular} 
	\caption{Grundlegende Information der 2. Laborübung}
\end{table}

Verwendete Messgeräte:
\begin{itemize}
	\item{A}
	\item{B}
\end{itemize}

Im Rahmen der 2. Laborübung sollten fünf unterschiedliche Impedanzen ($Z1$ - $Z5$) vermessen werden. Dabei war lediglich deren Struktur (siehe Tabelle XXXX) im Vorhinein bekannt. Es wurde zuerst ein passender Strommessshunt ausgewählt und die Schaltung konzipiert. Um aus den erhaltenen Spannungswerten den dazugehörigen Strom bestimmen zu können ist natürlich die genau Kenntnis über den Widerstandswert unabdingbar, weshalb dieser zu Beginn mehrmals und mit unterschiedlichen Methoden bestimmt worden ist. Die eigentliche Impedanzmessung wurde darauf hin mit einem analogen Oszilloskop durchgeführt. \\


\section{Widerstandsmessung}
\begin{table}[]
	\centering
	\begin{tabular}{|c|c|c|}
	\hline 
	Messung & Widerstandswert [$\Omega$] & Ergebnis \\ 
	\hline 
	M1 & 986 & • \\ 
	\hline 
	M2 & 984 & • \\ 
	\hline 
	M3 & 988 & • \\ 
	\hline 
	DM & 987 & • \\ 
	\hline 
	\end{tabular}
	\caption{•}
\end{table}


\begin{table}[]
	\centering
	\begin{tabular}{|c|c|c|c|}
	\hline 
	PLC & Samples & Mittelwert [$\Omega$] & Standardabweichung [m$\Omega$] \\ 
	\hline 
	$0.02$ & 15k & 987.44 & 20 \\ 
	\hline 
	$0.2$ & 15k & 987.421 & 12 \\ 
	\hline 
	$1$ & 273 & 987.442 & 3 \\ 
	\hline 
	• & 1017 & 987.416 & 2 \\ 
	\hline 
	• & 5456 & 987.401 & 13 \\ 
	\hline 
	\end{tabular}
	\caption{•}
\end{table}

\section{Impedanzmessung}
\begin{table}[]
	\centering
	\begin{tabular}{|c|c|c|c|c|c|}
	\hline 
	Strang & f [kHz] & u [V] & i [A] & Z [$\Omega$] & Struktur \\ 
	\hline 
	S1 & 1 & • & • & • & • \\ 
	\hline 
	• & 15 & • & • & • & • \\ 
	\hline 
	S2 & 1 & • & • & • & • \\ 
	\hline 
	• & 15 & • & • & • & • \\ 
	\hline 
	S3 & 1 & • & • & • & • \\ 
	\hline 
	• & 15 & • & • & • & • \\ 
	\hline 
	S4 & 1 & • & • & • & • \\ 
	\hline 
	• & 15 & • & • & • & • \\ 
	\hline 
	S5 & 1 & • & • & • & • \\ 
	\hline 
	• & 15 & • & • & • & • \\ 
	\hline 
	\end{tabular}
	\caption{•}
\end{table}
\section{Fehlerforpflanzung}
\begin{table}
	\centering
	\begin{tabular}{|c||c|c|}
	\hline 
	Messung Nr. & $x_1 = I_{RMS}$ [V] & $x_2 = \Phi$ [rad] \\ 
	\hline 
	1 & • & • \\ 
	\hline 
	2 & • & • \\ 
	\hline 
	3 & • & • \\ 
	\hline 
	4 & • & • \\ 
	\hline 
	5 & • & • \\ 
	\hline 
	6 & • & • \\ 
	\hline 
	$\overline{x_i}$ & • & • \\ 
	\hline 
	$s(\overline{x_i})$ & • & • \\ 
	\hline 
	$\frac{\partial P}{\partial x_i}$ & • & • \\ 
	\hline 
	$(\frac{\partial P}{\partial x_i})^2 s^2(\overline{x_i})$ & • & • \\ 
	\hline 
	Kovarianz & \multicolumn{2}{c|}{•} \\ 
	\hline 
	$s (\overline{P})$ & \multicolumn{2}{c|}{•} \\ 
	\hline 
	\end{tabular} 
	\caption{•}
\end{table}
\section{Impedazmessung mit LCR-Meter}
\begin{table}[]
	\centering
	\begin{tabular}{|c||c|c|c|}
	\hline 
	Strang & C/L [nF/mH] & R [$\Omega$] & Z [$\Omega$] \\ 
	\hline 
	S1 & 47.84 & 13.57 & 300.9 \\ 
	\hline 
	S2 & 1.1018 & 17.80 & 18.91 \\ 
	\hline 
	S3 & 97.89 & \num{2.701e+3} & \num{3.153e+3} \\ 
	\hline 
	S4 & 102 & 27.5 & \num{1.56e+3} \\ 
	\hline 
	S5 &  & \num{8.066} & \num{8.066e+3} \\ 
	\hline 
	\end{tabular} 
\end{table}
TO DO
\section{5/8-Methode}
TO DO
\section{Leistungsmessung}